\chapter*{Abstract}
\addcontentsline{toc}{chapter}{Abstract}
Nel panorama della sicurezza informatica i primi attacchi alla memoria, comunemente conosciuti come \textit{memory corruption attacks} risalgono agli anni 80 e sono stati sfruttati per la prima volta da Robert Tappan Morris \cite{FBI} che, individuando una vulnerabilità nel codice di un applicativo Unix, è riuscito a eseguire codice arbitrario manipolando lo stack di un programma C. Questo attacco oggi è conosciuto come \textit{Buffer Overflow} ed è solo l'inizio di una serie di attacchi che sarà di fondamentale importanza per la sicurezza informatica degli anni a venire. \\
Gli attacchi di quei tempi si concentravano principalmente su sistemi UNIX e prendevano di mira le classiche architetture x86 montate su processori come Intel 8086 \cite{Intel} o Intel 8088 dotate di istruzioni a lunghezza variabile. Dato che gli attacchi vengono fatti a tempo di esecuzione e coinvolgono strettamente lo stack di esecuzione del programma, si può pensare che siano strettamente legati al tipo di architettura sottostante e dipendano dalle scelte implementative degli ingegneri che hanno progettato l'ISA. In questa tesi si analizzerà questo aspetto, chiarendo quali elementi sono dipendenti dall'architettura durante gli attacchi di Buffer Overflow e Return Oriented Programming.\\
In questa tesi ho analizzato quindi la differenza tra gli attacchi di architetture x86\_64 bit, ovvero la versione a 64 bit della tradizionale architettura x86 e gli attacchi ai binari basati sul moderno standard RISC-V, la quinta versione dell'ISA RISC, open source, con istruzioni a lunghezza fissa, moderna e governata dalla fondazione RISC-V International. Questa analisi è importante dato il grande successo che sta avendo RISC-V che in pochi anni visto il basso costo implementativo è sempre più presente in microprocessori e system on a chip di vario tipo. Al giorno d'oggi è facile trovare implementazioni di ISA RISC-V anche su sistemi embedded e macchinari automatici grazie all'alto grado di specializzazione permesso dall'ISA open source ed ai bassi consumi che è possibile raggiungere su processori che implementano questa ISA. \\
È quindi cruciale essere consapevoli delle vulnerabilità presenti nell'architettura e negli eseguibili prodotti da processori RISC-V che, pur essendo open source, lascia grande spazio all'ingegnere che deve essere in primo luogo attento al codice che lui stesso produce. \\
In questa tesi ho analizzato i tipi di attacco che è possibile riprodurre sulle due architetture, le differenze, i limiti di sicurezza e gli scenari in cui è più vantaggioso cercare di attaccare una o l'altra architettura. Ho contribuito inoltre fornendo dei codici sorgenti e delle fingerprint di eseguibili per RISC-V volutamente vulnerabili e attaccabili con Buffer Overflow e Return Oriented Programming. Queste tracce potranno essere poi fornite a dei sistemi che applicano Control Flow Integrity per studiare se minacce di questo tipo possono venire rilevate.\\
Ogni codice, artefatto e test eseguito è tracciato nelle seguenti repository:\\
\begin{itemize}
    \item \href{https://github.com/BlessedRebuS/RISCV-Attacks}{\textbf{https://github.com/BlessedRebuS/RISCV-Attacks}}
    \item \href{https://github.com/BlessedRebuS/RISCV-ROP-Testbed}{\textbf{https://github.com/BlessedRebuS/RISCV-ROP-Testbed}}
\end{itemize}
\newpage
\newpage